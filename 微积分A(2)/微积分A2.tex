\documentclass[UTF8,openany]{book}
\usepackage{amsmath}
\usepackage{amssymb}
\usepackage{ctex}
\usepackage{esint}
\begin{document}
	\chapter{常数项级数}

	\chapter{函数项级数}

	\chapter{n维Euclid空间}

	\chapter{多元函数微分学}
	\section{多元函数的定义与极限}
	\subsection{多元函数的定义}
	\par 设$m,n\leqslant1$为整数,$\Omega\subseteq\mathbb{R}^n$为非空集合,定义映射$\textbf{\textit{f}}:\Omega\rightarrow\mathbb{R}^m$
	为定义在$\Omega$上的\textbf{n元向量值函数},当$n=1$时,$f$可以简称为\textbf{n元函数}。
	设$X=(x_1,\cdots,x_n)$,$Y=(y_1,\cdots,y_m)$,则对于$\forall i, 1\leqslant i\leqslant m$,则
	$y_i=f_i(X)$,因此n元向量值函数等价于m个n元函数,也就是
	\[
	\textbf{\textit{f}}=(f_1,\cdots,f_m)	
	\]
	\par 与一元函数类似,多元向量值函数也可以进行一些运算。
	$$(\lambda\textbf{\textit{f}}+\mu\textbf{\textit{g}})(X)=\lambda\textbf{\textit{f}}(X)+
	\mu\textbf{\textit{g}}(X)$$
	$$(g\textbf{\textit{f}})(X)=g(X)\textbf{\textit{f}}(X)$$
	$$(\frac{\textbf{\textit{f}}}{g})(X)=\frac{\textbf{\textit{f}}(X)}{g(X)}$$
	$$(\textbf{\textit{g}}\circ\textbf{\textit{f}})(X)=\textbf{\textit{g}}(\textbf{\textit{f}}(X))$$
	\subsection{多元函数的极限}
	\par 设$\Omega\subseteq\mathbb{R}^n$,$\textbf{\textit{f}}:\Omega\rightarrow\mathbb{R}^m$,
	$X_0\in\mathbb{R}^n$是$\Omega$的极限点,$A\in\mathbb{R}^m$。
	若对于$\forall\varepsilon>0$,$\exists\delta>0$,使得$\forall X\in\Omega$,当$0<||X-X_0||_n<\delta$时,
	有$||\textbf{\textit{f}}(X)-A||_m<\varepsilon$;或者说$\forall X\in\mathring{B}(X_0,\delta)\cap\Omega$,
	都有$\textbf{\textit{f}}(X)\in B(A,\varepsilon)$,则称当$X$趋于$X_0$时,$\textbf{\textit{f}}(X)$收敛于$A$,
	以$A$为极限,记作
	$$\lim_{\Omega\ni X\rightarrow X_0}\textbf{\textit{f}}(X)=A$$
	\par 如果$X_0$是$\Omega\cup\{X_0\}$的内点,极限符号也可以简记为
	$$\lim_{X\rightarrow X_0}\textbf{\textit{f}}(X)=A$$
	\par n元向量值函数的极限具有下面的性质:
	\par (1)记$\textbf{\textit{f}}=(f_1,\cdots,f_m)$,$A=(a_1,\cdots,a_m)$,则对于$\forall 1\leqslant i\leqslant m$,都有
	$$\lim_{\Omega\ni X\rightarrow X_0}f_i(X)=a_i$$
	\par (2)同一元函数极限一样,多元向量值函数的极限也具有唯一性、四则运算(加、减、(数)乘)、复合运算、序列、Cauchy定理等。
	\par (3)对于多元函数,由于可以比较大小,因此具有保序性、保号性、夹逼定理等。
	\par (4)对于二重极限,假如$f(x,y)$在$(x,y)\rightarrow(x_0,y_0)$时极限存在,且在$x_0$某去心邻域内,$f(x,y)$关于$y$的极限存在,则有
	$$\lim_{(x,y)\rightarrow(x_0,y_0)}f(x,y)=\lim_{y\rightarrow y_0}\lim_{x\rightarrow x_0}f(x,y)$$
	\par 由此可以得到推论:若二重极限与某一个累次极限存在,则二者必然相等;若两个累次极限存在而不等,则二重极限不存在。
	\section{多元函数的连续性}
	\subsection{多元函数连续性的定义}
	\par 设$\textbf{\textit{f}}:\mathbb{R}^n\supseteq\Omega\rightarrow\mathbb{R}^m$,$X_0\in\Omega$,若
	$$\lim_{\Omega\ni X\rightarrow X_0}\textbf{\textit{f}}(X)=\textbf{\textit{{f}}}(X_0)$$
	则称函数$\textbf{\textit{f}}$在$X_0$处连续。若$\textbf{\textit{f}}$在$\Omega$上任一点都连续,则称$\textbf{\textit{f}}$在$\Omega$
	上连续,记为$\textbf{\textit{f}}\in C(\Omega;\mathbb{R}^m)$。
	\par 连续函数有下面的简易性质:
	\par (1)连续函数进行加、减、(数)乘、复合运算后结果仍为连续函数。
	\par (2)设$\textbf{\textit{f}}:\mathbb{R}^n\supseteq\Omega\rightarrow\mathbb{R}^m$,则$\textbf{\textit{f}}$连续当且仅当
	对$\mathbb{R}^m$任意开(闭)集$G$,其原像集$\textbf{\textit{f}}^{-1}(G)$也是开(闭)集。
	\subsection{最值定理与介值定理}
	\par 由于涉及到大小的比较,下面的函数都是针对多元函数而言的。
	\par \textbf{最值定理} \quad
	设$\Omega\subseteq\mathbb{R}^n$为闭集合,$f\in C(\Omega)$,则$f$在$\Omega$上有最大值和最小值。
	\par \textbf{介值定理} \quad
	设$\Omega\subseteq\mathbb{R}^n$为弧连通集合(集合内任意两点都可以用弧线连接),$f\in C(\Omega)$,则$\forall X_1,X_2\in\Omega$,
	以及$\forall\mu$介于$f(X_1)$与$f(X_2)$之间,总$\exists X_0\in\Omega$,使得$f(X_0)=\mu$。
	\section{多元函数的导数}
	\subsection{偏导数}
	\par \textbf{多元函数}
	\par 假设$f:\mathbb{R}^n\supseteq\Omega\rightarrow\mathbb{R}$,$\mathbb{R}$的基底为$\{\textbf{\textit{e}}_1,\cdots\textbf{\textit{e}}_n\}$,
	$f$在$X_0$的一个邻域内有定义,设$1\leqslant i\leqslant n$如果极限
	\[
	\lim_{h\rightarrow 0}\frac{f(X_0+h\textbf{\textit{e}}_i)-f(X_0)}{h}	
	\]
	存在,则我们将上面表达式的值称为$f$在$X_0$关于变量$x_i$的\textbf{偏导数},则积分可以记作
	\[
	\frac{\partial f}{\partial x_i}(X_0)\quad\text{或者}\quad\partial_{x_i}f(X_0)	
	\]
	\par 如果$f$在点$X_0$处关于所有变量的偏导数均存在,则称$f$在$X_0$是\textbf{可导的}。
	\par 采取行向量、列向量的表示方法,如果$f$是可微的,则可以表示为如下形式
	\[
	\mathrm{d}f=\left(\frac{\partial f}{\partial x_1},\cdots,\frac{\partial f}{\partial x_n}\right)\mathrm{d}X
	=\left(\frac{\partial f}{\partial x_1},\cdots,\frac{\partial f}{\partial x_n}\right)
	\begin{pmatrix}
		\mathrm{d}x_1\\
		\vdots\\
		\mathrm{d}x_n\\
	\end{pmatrix}
	\]
	\par \textbf{向量值函数}
	\par 现在我们考虑$\textbf{\textit{f}}:\mathbb{R}^n\supseteq\Omega\rightarrow\mathbb{R}^m$,则考虑对各个分量的偏导数即可。
	下面考虑其微分表达形式。我们有
	\[
	\begin{pmatrix}
		\mathrm{d}f_1\\
		\vdots\\
		\mathrm{d}f_m\\
	\end{pmatrix}
	=
	\begin{pmatrix}
		\partial_{x_1}f_1 & \ldots & \partial_{x_n}f_1 \\
		\vdots & & \vdots \\
		\partial_{x_1}f_m & \ldots & \partial_{x_n}f_m \\
	\end{pmatrix}
	\begin{pmatrix}
		\mathrm{d}x_1\\
		\vdots\\
		\mathrm{d}x_n\\
	\end{pmatrix}
	\]
	\par 我们将$(\partial_{x_j}f_i)_{m\times n}$称为$\textbf{\textit{f}}$的Jacobi矩阵,记作
	\[
	\textbf{\textit{J}}_{\textbf{\textit{f}}}\quad\text{或者}\quad
	\frac{\partial(f_1,\cdots,f_m)}{\partial(x_1,\cdots,x_n)}	
	\]
	\par 这样函数$\textbf{\textit{f}}$的微分就可以表示为
	\[
	\mathrm{d}\textbf{\textit{f}}=\textbf{\textit{J}}_{\textbf{\textit{f}}}\mathrm{d}X
	\]
	\par 特别地,如果$m=n$,此时Jacobi矩阵是方阵,因而有行列式,称为Jacobi行列式,记作
	\[
	\frac{D(f_1,\cdots,f_n)}{D(x_1,\cdots,x_n)}=
	\det\textbf{\textit{J}}_{\textbf{\textit{f}}}=
	\det\frac{\partial(f_1,\cdots,f_m)}{\partial(x_1,\cdots,x_n)}
	\]
	\subsection{高阶导数}
	\par 如果某一个偏导数是可导的,就可以再次求导,得到\textbf{高阶偏导数}。
	\[
	\frac{\partial}{\partial x_j}\left(\frac{\partial f}{\partial x_i}\right)=
	\frac{\partial^2f}{\partial x_j\partial x_i}	
	\]
	\[
	\frac{\partial}{\partial x_i}\left(\frac{\partial f}{\partial x_i}\right)=
	\frac{\partial^2f}{\partial x_i^2}	
	\]
	\par 对于函数$f$,先对$x_i$求导,再对$x_j$求导,可以得到一个偏导数;如果交换求导顺序,可以得到另一个偏导数。如果两个偏导数
	在定义域内的$X_0$点连续,则有
	\[
	\frac{\partial^2 f}{\partial x_j\partial x_i}(X_0)=\frac{\partial^2 f}{\partial x_i\partial x_j}(X_0)	
	\]
	\subsection{复合函数、隐函数、反函数的求导}
	\par \textbf{复合函数}
	\par 设$\textbf{\textit{f}}$与$\textbf{\textit{g}}$可以构成复合函数$\textbf{\textit{g}}\circ\textbf{\textit{f}}$,我们设
	\[
	\mathrm{d}\textbf{\textit{f}}=\textbf{\textit{J}}_\textbf{\textit{f}}\mathrm{d}X	
	\]
	\[
	\mathrm{d}\textbf{\textit{g}}=\textbf{\textit{J}}_\textbf{\textit{g}}\mathrm{d}Y
	\]
	则有
	\[
	\mathrm{d}(\textbf{\textit{g}}\circ\textbf{\textit{f}})=\textbf{\textit{J}}_\textbf{\textit{g}}
	\textbf{\textit{J}}_\textbf{\textit{f}}\mathrm{d}X	
	\]
	\par 也就是说
	\(
	\textbf{\textit{J}}_{\textbf{\textit{g}}\circ\textbf{\textit{f}}}=
	\textbf{\textit{J}}_\textbf{\textit{g}}
	\textbf{\textit{J}}_\textbf{\textit{f}}	
	\)
	\par \textbf{反函数}
	\par 根据
	\[
	\mathrm{d}Y=\textbf{\textit{J}}_{\textbf{\textit{f}}}\mathrm{d}X
	\]
	\par 则
	\[
	\mathrm{d}Y=\textbf{\textit{J}}_{\textbf{\textit{f}}}^{-1}\mathrm{d}Y	
	\]
	\par 也就是说
	\[
	\textbf{\textit{J}}_{\textbf{\textit{f}}^{-1}}=\textbf{\textit{J}}_{\textbf{\textit{f}}}^{-1}	
	\]
	\par \textbf{隐函数}
	\par 首先考虑从$F(X,y)=0$中解出$y=f(X)$。
	\par 设$X_0\in\mathbb{R}^n$,$y_0\in\mathbb{R}$。给定函数$F:B((X_0,y_0),r)\rightarrow\mathbb{R}
	\in C^{(1)}$,满足$F(X_0,y_0)=0$,且$\partial_y F(X_0,y_0)\neq 0$。那么,就存在从$X$到$y$的隐函数。也就是说,
	$\exists\delta,\eta$
	\[
	B(X_0,\delta)\times B(y_0,\eta)\subset B((X_0,y_0),r)	
	\]
	对于$\forall X\in B(X_0,\delta)$,$\exists!y\in B(y_0,\eta)$,使得$F(X,y)=0$。这样我们就找到了
	从$X$到$y$的函数$f:B(X_0,\delta)\rightarrow B(y_0,\eta)\in C^{(1)}$,且
	\[
	\frac{\partial f}{\partial x_i}=-\frac{\partial_{x_i}F}{\partial_{y}F}	
	\]
	\par 下面考虑从$\textbf{\textit{F}}(X,Y)=0$解出$Y=\textbf{\textit{f}}(X)$。
	\par 设$X_0\in\mathbb{R}^n$,$Y_0\in\mathbb{R}^m$。给定函数$\textbf{\textit{F}}:B((X_0,Y_0),r)\rightarrow
	\mathrm{R}\in C^{(1)}$,满足$\textbf{\textit{F}}(X_0,Y_0)=0$,且$\det\textbf{\textit{J}}_
	\textbf{\textit{y}}\textbf{\textit{F}}\neq 0$。那么,就存在从$X$到$Y$的隐函数。也就是说,$\exists\delta,\eta$
	\[
	B(X_0,\delta)\times B(Y_0,\eta)\subset B((X_0,y_0),r)	
	\]
	对于$\forall X\in B(X_0,\delta)$,$\exists!Y\in B(Y_0,\eta)$,使得$\textbf{\textit{F}}(X,Y)=0$。这样我们就找到了
	从$X$到$Y$的函数$\textbf{\textit{f}}:B(X_0,\delta)\rightarrow B(y_0,\eta)\in C^{(1)}$,且
	\[
	\frac{\partial(f_1,\cdots,f_m)}{\partial(x_1,\cdots,x_n)}=-
	\left(\frac{\partial(F_1,\cdots,F_m)}{\partial(y_1,\cdots,y_m)}\right)^{-1}
	\frac{\partial(F_1,\cdots,F_m)}{\partial(x_1,\cdots,x_n)}	
	\]
	\[
	\textbf{\textit{J}}_{\textbf{\textit{x}}}\textbf{\textit{f}}=-
	\frac{
	\textbf{\textit{J}}_{\textbf{\textit{x}}}\textbf{\textit{F}}}
	{\textbf{\textit{J}}_{\textbf{\textit{y}}}\textbf{\textit{F}}}
	\]
	\section{多元函数的微分}
	















	\chapter{含参积分}
	\section{含参积分的定义与性质}
	\subsection{含参积分的定义}
	\par 在函数连续的基础上,我们定义函数的一致连续性。设$f:\Omega\subset\mathbb{R}^n\rightarrow\mathbb{R}$,
	若对于$\forall\varepsilon>0$,$\exists\delta>0$,使得$\forall X,Y\in\Omega$,当$||X-Y||_n<\delta$时,
	都有$|f(X)-f(Y)|<\varepsilon$,则称$f$在$\Omega$上是\textbf{一致连续}的。
	\par 一致连续的函数必然连续,连续函数未必一致连续。但是闭集上的连续函数必然是一致连续的。
	\par 定义函数$f:[a,b]\times[c,d]\rightarrow\mathbb{R}$,如果对于$\forall y\in[c,d]$
	$$I(y)=\int_{a}^{b}f(x,y)\mathrm{d}x$$
	有定义,则将$I(y)$称为以$y$为变量的\textbf{含参积分}。
	\subsection{含参积分的性质}
	\par \textbf{积分与极限可交换性}\quad 如果$f$是连续函数,则$I$也是连续函数,也就是说
	$$\lim_{y\rightarrow y_0}\int_{a}^{b}f(x,y)\mathrm{d}x=\int_{a}^{b}\lim_{y\rightarrow y_0}f(x,y)\mathrm{d}x$$
	如果将$[c,d]$换成$(c,d)$,结论依然成立。
	\par \textbf{积分与导数可交换性}\quad 如果$f$连续可导,则$I$也是连续可导的,并且
	$$I'(y)=\frac{\mathrm{d}}{\mathrm{d}y}\int_{a}^{b}f(x,y)\mathrm{d}x=\int_{a}^{b}\frac{\partial f}{\partial y}(x,y)\mathrm{d}x$$
	如果将$[c,d]$换成$(c,d)$,结论依然成立。
	\par 设$\alpha,\beta:[c,d]\rightarrow[a,b]$可导,则定义
	$$J(y)=\int_{\alpha(y)}^{\beta(y)}f(x,y)\mathrm{d}x$$
	我们有
	$$J'(y)=\int_{\alpha(y)}^{\beta(y)}\frac{\partial f}{\partial y}(x,y)\mathrm{d}x+f(\beta(y),y)\beta'(y)-f(\alpha(y),y)\alpha'(y)$$
	\par \textbf{积分与积分可交换性}\quad 如果$f$连续,我们有
	$$\int_{c}^{d}\mathrm{d}y\int_{a}^{b}f(x,y)\mathrm{d}x=\int_{a}^{b}\mathrm{d}x\int_{c}^{d}f(x,y)\mathrm{d}y$$
	\section{广义含参积分}
	\subsection{广义含参积分的定义}
	\par 假设$f:[a,w)\times[c,d]\rightarrow\mathbb{R}$为连续函数,其中$w\in\mathbb{R}_{\infty}$。若$y_0
	\in[c,d]$使得广义积分
	$$\int_{a}^{w}f(x,y_0)\mathrm{d}x=\lim_{A\rightarrow w^-}\int_{a}^{A}f(x,y_0)\mathrm{d}x$$
	收敛,则称广义积分在$y_0$处\textbf{收敛},否则在该点\textbf{发散}。
	\par 如果广义积分在$[c,d]$上每一点都收敛,我们就得到了函数
	$$I(y)=\int_{a}^{w}f(x,y)\mathrm{d}x$$
	\par 如果$\forall\varepsilon>0$,$\exists A_0\in[a,w)$,使得$\forall A\in[A_0,w)$以及$\forall y\in[c,d]$,
	均有
	\[
	\left|\int_{a}^{A}f(x,y)\mathrm{d}x-I(y)\right|<\varepsilon	
	\]
	则称广义积分在区间$[c,d]$上\textbf{一致收敛}到函数$I(y)$。
	\subsection{广义积分收敛性质的判断}
	\par\textbf{Cauchy准则}\quad
	广义积分在$[c,d]$上一致收敛当且仅当$\forall\varepsilon>0$,$\exists A_0\in[a,w)$
	使得$\forall A',A''\in[A_0,w)$,$\forall y\in[c,d]$,都有
	\[
	\left|\int_{A'}^{A''}f(x,y)\mathrm{d}x\right|<\varepsilon
	\]
	\par\textbf{Weierstrass判别法}\quad
	假设$f:[a,w)\times[c,d]\rightarrow\mathbb{R}$为连续函数,函数$F:[a,w)\rightarrow[0,+\infty)$
	使得对于$\forall(x,y)\in[a,w)\times[c,d]$都有$|f(x,y)|\leqslant F(x)$,如果积分
	\[
	\int_{a}^{w}F(x)\mathrm{d}x	
	\]
	收敛,则
	\[
	\int_{a}^{w}f(x,y)\mathrm{d}x	
	\]
	关于$y\in[c,d]$一致收敛。
	\par\textbf{Abel判别法}\quad
	设$f,g:[a,w)\times[c,d]\rightarrow\mathbb{R}$,对于$\forall y\in[c,d]$,
	$f(x,y)$和$g(x,y)$个在$[a,w)$任意闭子区间都可积,如果$f(x,y)$的广义积分关于$y\in[c,d]$一致收敛,
	而$g$有界并且关于$x$单调,则
	\[
	\int_a^wf(x,y)g(x,y)\mathrm{d}x	
	\]
	关于$y\in[c,d]$一致收敛。
	\par\textbf{Dirichlet判别法}\quad
	设$f,g$,其可积性与上一段的叙述相同。对于$\forall y\in[c,d]$以及
	$\forall A\in[a,w)$,定义
	\[
	F(A,y)=\int_{a}^{A}f(x,y)\mathrm{d}x	
	\]
	\par 如果$F$有界,$g$关于$x$单调,且当$x\rightarrow w^-$时$g(x,y)\rightarrow 0$一致成立,则
	\[
	\int_{a}^{w}f(x,y)g(x,y)\mathrm{d}x	
	\]
	一致收敛。
	\subsection{广义含参积分的性质}
	\par\textbf{极限与积分可交换性}\quad
	设$f:[a,w)\times[c,d]\rightarrow\mathbb{R}\in C$,若广义积分
	\[
	I(y)=\int_{a}^{w}f(x,y)\mathrm{d}x	
	\]关于$y\in[c,d]$一致收敛,则$I$在$[c,d]$上连续。将$[c,d]$换成开区间也成立。
	\par\textbf{求导与积分可交换性}\quad
	设$f\in C$,若$I(y)$在$[c,d]$收敛,$\partial_yf\in C$且相应广义含参积分一致收敛,则
	$I$在$[c,d]$上连续可导,并且
	\[
	I'(y)=\int_{a}^{w}\frac{\partial f}{\partial y}(x,y)\mathrm{d}x	
	\]
	\par\textbf{积分与积分可交换性}\quad
	设$f\in C$,若$I(y)$一致收敛,则$I\in R[c,d]$,且
	\[
	\int_{c}^{d}\mathrm{d}y\int_{a}^{w}f(x,y)\mathrm{d}x=\int_{a}^{w}\mathrm{d}x\int_{c}^{d}f(x,y)\mathrm{d}y	
	\]
	\section{Gamma函数与Beta函数}
	\subsection{Gamma函数}
	\par Gamma函数的定义为
	\[
	\Gamma(s)=\int_{0}^{+\infty}x^{s-1}e^{-x}\mathrm{d}x	
	\]
	\par 通过变量代换,可以得到Gamma函数的其他表示方式,例如
	\[
	\Gamma(s)=2\int_{0}^{\infty}x^{2s-1}e^{-x^2}\mathrm{d}x\quad,s>0	
	\]
	\[
	\Gamma(s)=a^s\int_{0}^{\infty}x^{s-1}e^{-ax}\mathrm{d}x\quad,s>0,a>0	
	\]
	\par Gamma函数有下面的性质:
	\par (1)$\Gamma(s)$在$(0,+\infty)$上收敛且连续,并且有任意阶连续导数。
	\par (2)$\forall s>0$,$\Gamma(s)>0$,且$\Gamma(1)=1$,$\Gamma(\frac{1}{2})=\sqrt{\pi}$。
	\par (3)$\forall s>0$,$\Gamma(s+1)=s\Gamma(s)$,若$n$是自然数,则$\Gamma(n+1)=n!$
	\par (4)$\ln\Gamma(s)$是$(0,+\infty)$上的凸函数。
	\par (5)定义在$(0,\infty)$上的函数$f(s)$若满足性质(2)—(4),则$f$必然是Gamma函数。
	\par (6)(余元公式)对于$\forall p\in(0,1)$,有
	$$\Gamma(p)\Gamma(1-p)=\frac{\pi}{\sin p\pi}$$
	\subsection{Beta函数}
	\par Beta函数的定义为
	\[
	B(p,q)=\int_{0}^{1}x^{p-1}(1-x)^{q-1}\mathrm{d}x	
	\]
	\par 通过变量代换,可以得到Beta函数的其他表示方式,例如用三角函数,以及下面的
	\[
	B(p,q)=\int_{0}^{\infty}\frac{t^{q-1}}{(1+t)^{p+q}}\mathrm{d}t\quad,p>0,q>0
	\]
	\par Beta函数有下面的性质
	\par (1)$B(p,q)$在$(0,+\infty)\times(0,+\infty)$上收敛、连续,并且有各阶连续偏导数。
	\par (2)$\forall(p,q)\in(0,+\infty)\times(0,+\infty)$,都有$B(p,q)=B(q,p)$。
	\par (3)$$B(p+1,q+1)=\frac{pq}{(p+q+1)(p+q)}B(p,q)$$。
	\par (4)$$B(p,q)=\frac{\Gamma(p)\Gamma(q)}{\Gamma(p+q)}$$

	\chapter{重积分}
	\section{重积分的定义与性质}
	\subsection{重积分的定义}
	首先,我们考虑最理想的情形,积分区域是一个立方体。我们定义$n$维平行体
	$$I=\{(x_1,\cdots,x_n)\in\mathbb{R}^n|a_i\leqslant x_i\leqslant b_i,\quad1\leqslant i\leqslant n\}$$
	并定义其体积
	$$|I|=\prod_{i=1}^{n}(b_i-a_i)$$
	以及上面的函数
	$$f:I\rightarrow \mathbb{R}$$
	\par 现在,将其分割成一系列更小的立方体,得到的一系列立方体构成$I$的一个分割
	$$P=\{I_i\}_{i=1}^{k}$$
	从每一个小的立方体上选取一个点$\boldsymbol{\xi}_i\in I_i$,对应一个函数值$f(\boldsymbol{\xi}_i)$,定义Riemann和
	$$\sum_{i=1}^{k}f(\boldsymbol{\xi}_i)|I_i|$$
	\par 定义分割$P$的步长为
	$$\lambda(P)=\max_{J\in P}d(J)$$
	\par 如果极限
	$$A=\lim_{\lambda(P)\rightarrow0}\sum_{i=1}^{k}f(\boldsymbol{\xi}_i)|I_i|$$
	存在,我们称$f$在$I$上\textbf{Riemann可积},将$A$称为$f$在$I$上的\textbf{n重积分},记作
	$$\int\limits_{I}f(\textbf{\textit{X}})\mathrm{d}\textbf{\textit{X}}$$
	或者
	$$\int\cdots\int_{I} f(x_1,\cdots,x_n)\mathrm{d}x_1\cdots\mathrm{d}x_n$$
	\par 下面考虑一般的积分区域,假设$\Omega\in I$,可以定义
	$$\tilde{f}(\textbf{\textit{X}})=
	\begin{cases}
	f(\textbf{\textit{X}}), & \qquad \textbf{\textit{X}}\in\Omega\\
	0, & \qquad \textbf{\textit{X}}\in I\textbackslash\Omega
	\end{cases}$$
	\par 如果$\tilde{f}(X)$在$I$上Riemann可积,则$f(X)$在$\Omega$上Riemann可积,并且我们有
	$$\int\limits_{\Omega}f(\textbf{\textit{X}})\mathrm{d}\textbf{\textit{X}}=
	\int\limits_{I}\tilde{f}(\textbf{\textit{X}})\mathrm{d}\textbf{\textit{X}}$$
	\par 下面则是一些有关测度的内容。定义$\Omega$上的示性函数$I:\Omega\rightarrow\{0,1\}$如下:
	$$I(\textbf{\textit{X}})=\begin{cases}
	1, & \qquad \textbf{\textit{X}}\in\Omega\\
	0, & \qquad \textbf{\textit{X}}\notin\Omega\\
	\end{cases}$$
	\par 如果$\Omega$是有界集并且使得$I(\textbf{\textit{X}})$Riemann可积,我们称$\Omega$是\textbf{Jordan可测集},并定义集合$\Omega$的测度为
	$$|\Omega|=\int\limits_{\Omega}I(\textbf{\textit{X}})\mathrm{d}\textbf{\textit{X}}$$
	\par 如果有界闭集$\Omega$是Jordan可测集,且定义在上面的$f$连续,则$f$在$\Omega$上Riemann可积。$\Omega$上Riemann可积函数的全体记作$R(\Omega)$。
	\subsection{重积分的性质}
	首先是一些关于被积函数的性质。下面的性质都是针对Jordan可测集而言的。
	\par \textbf{有界性}\quad 若$f\in R(\Omega)$,则$f$有界。
	\par \textbf{线性}\quad 若$f_1,f_2\in R(\Omega)$,$a,b\in\mathbb{R}$,则$af_1+bf_2\in R(\Omega)$,并且
	$$\int\limits_{\Omega}(af_1+bf_2)\mathrm{d}\textbf{\textit{X}}=
	a\int\limits_{\Omega}f_1\mathrm{d}\textbf{\textit{X}}+b\int\limits_{\Omega}f_2\mathrm{d}\textbf{\textit{X}}$$
	\par \textbf{保号性}\quad 若$\forall \textbf{\textit{X}}\in\Omega$,有$f(\textbf{\textit{X}})\geqslant0$,则
	$$\int\limits_{\Omega}f(\textbf{\textit{X}})\mathrm{d}\textbf{\textit{X}}\geqslant0$$
	\par \textbf{严格保号性}\quad 若$\forall\textbf{\textit{X}}\in\Omega$,$f(\textbf{\textit{X}})\geqslant0$,
	而$\exists\textbf{\textit{X}}_0\in\Omega$使得$f(\textbf{\textit{X}}_0)\neq0$,则
	$$\int\limits_{\Omega}f(\textbf{\textit{X}})\mathrm{d}\textbf{\textit{X}}>0$$
	\par \textbf{保序性}\quad 若$\forall\textbf{\textit{X}}\in\Omega$,都有$f(\textbf{\textit{X}})\geqslant g(\textbf{\textit{X}})$,则
	$$\int\limits_{\Omega}f(\textbf{\textit{X}})\mathrm{d}\textbf{\textit{X}}\geqslant
	\int\limits_{\Omega}g(\textbf{\textit{X}})\mathrm{d}\textbf{\textit{X}}$$
	\par \textbf{绝对值不等式}\quad 如果$f\in R(\Omega)$,则$|f|\in\Omega$,且
	$$\left|\int\limits_{\Omega}f(\textbf{\textit{X}})\mathrm{d}\textbf{\textit{X}}\right|\leqslant
	\int\limits_{\Omega}|f(\textbf{\textit{X}})|\mathrm{d}\textbf{\textit{X}}$$
	\par \textbf{上下界估计}\quad 如果$M=\sup(f)$,$N=\inf(f)$则
	$$m|\Omega|\leqslant\int\limits_{\Omega}f(\textbf{\textit{X}})\mathrm{d}\textbf{\textit{X}}\leqslant M|\Omega|$$
	\par \textbf{中值定理}\quad 如果$\Omega$是有界闭连通集合,则$\exists\textbf{\textit{X}}_0\in\Omega$,使得
	$$\int\limits_{\Omega}f(\textbf{\textit{X}})\mathrm{d}\textbf{\textit{X}}=f(\textbf{\textit{X}}_0)|\Omega|$$
	由此可以得到推论,对于$\forall\textbf{\textit{Y}}\in\mathring{\Omega}$,我们有
	$$f(\textbf{\textit{Y}})=\lim_{r\rightarrow0^+}\frac{1}{|\bar{B}(\textbf{\textit{Y}},r)|}\int_{\bar{B}(\textbf{\textit{Y}},r)}f(\textbf{\textit{X}})\mathrm{d}\textbf{\textit{X}}$$
	\par 下面是一些关于积分区域的性质。下面的性质都是针对Jordan可测集而言的。
	\par \textbf{区域可加性}\quad 如果$\Omega_1,\Omega_2\subset\Omega$是Jordan可测集,
	$\Omega=\Omega_1\cup\Omega_2$,且$\Omega_1$与$\Omega_2$没有公共内点,则$f\in R(\Omega)$当且仅当$f\in R(\Omega_1)$且$f\in R(\Omega_2)$。此时有
	$$\int\limits_{\Omega}f(\textbf{\textit{X}})\mathrm{d}\textbf{\textit{X}}=
	\int\limits_{\Omega_1}f(\textbf{\textit{X}})\mathrm{d}\textbf{\textit{X}}+
	\int\limits_{\Omega_2}f(\textbf{\textit{X}})\mathrm{d}\textbf{\textit{X}}$$
	\par \textbf{坐标变换}\quad 若$\Omega_1,\Omega_2\in\mathbb{R}^n$,其中$\Omega_1=\{(x_1,\cdots,x_n)\}$,
	$\Omega_2=\{(y_1,\cdots,y_n)\}$,$\boldsymbol{\varphi}=(\varphi_1,\cdots,\varphi_n)$是将$\Omega_1$变换到$\Omega_2$的双射,即$\textbf{\textit{Y}}=\boldsymbol{\varphi}(\textbf{\textit{X}})$,也就是
	$$\begin{cases}
	y_1=\varphi_1(\textbf{\textit{X}})\\
	\cdots\\
	y_n=\varphi_n(\textbf{\textit{X}})\\
	\end{cases}$$
	\par 设$D_1\subset\Omega_1$,经过变换后得到$D_2=\boldsymbol{\varphi}(D_1)\subset\Omega_2$。如果$\boldsymbol{\varphi}$和其逆映射$\boldsymbol{\varphi}^{-1}$都是连续可导的,$D_1$是Jordan可测集,则$D_2$也是Jordan可测集,并且对$\forall f\in C$,有
	$$\int\limits_{D_2}f(\textbf{\textit{Y}})\mathrm{d}\textbf{\textit{Y}}=
	\int\limits_{D_1}f(\boldsymbol{\varphi}(\textbf{\textit{X}}))
	\left|\frac{D(\varphi_1,\cdots,\varphi_n)}{D(x_1,\cdots,x_n)}\right|\mathrm{d}\textbf{\textit{X}}$$
	\section{二重积分的计算}
	\subsection{二重积分的一般计算方法}
	\textbf{X型区域}\quad 将积分区域记为$\Omega$,设$f_1$,$f_2$在$[a,b]$上连续,且$\forall x\in[a,b]$,都有$f_1(x)\leqslant f_2(x)$,则区域可以表示为
	$$\Omega=\{(x,y)|a\leqslant x\leqslant b,\quad f_1(x)\leqslant y\leqslant f_2(x)\}$$
	\par 这个区域是Jordan可测的。如果$f:\Omega\rightarrow\mathbb{R}$是连续函数,则
	$$\iint\limits_{\Omega}f(x,y)\mathrm{d}x\mathrm{d}y=
	\int_{a}^{b}\mathrm{d}x\int_{f_1(x)}^{f_2(x)}f(x,y)\mathrm{d}y$$
	\par \textbf{Y型区域}\quad 将积分区域记为$\Omega$,设$g_1$,$g_2$在$[c,d]$上连续,且$\forall y\in[c,d]$,都有$g_1(y)\leqslant g_2(y)$,则区域可以表示为
	$$\Omega=\{(x,y)|c\leqslant y\leqslant d,\quad g_1(y)\leqslant x\leqslant g_2(y)\}$$
	\par 这个区域是Jordan可测的。如果$f:\Omega\rightarrow\mathbb{R}$是连续函数,则
	$$\iint\limits_{\Omega}f(x,y)\mathrm{d}x\mathrm{d}y=
	\int_{c}^{d}\mathrm{d}y\int_{g_1(y)}^{g_2(y)}f(x,y)\mathrm{d}x$$
	\subsection{极坐标变换}
	如果积分区域与圆有关,或者积分表达式中涉及$x^2+y^2$之类的项,可以考虑极坐标变换。设$D_1=\{(\rho,\varphi)\}$,$D_2=\{(x,y)\}$,$\boldsymbol{\varphi}:D_1\rightarrow D_2$,则
	$$\begin{cases}
	x=\varphi_1(\rho,\varphi)=\rho\cos\varphi\\
	y=\varphi_2(\rho,\varphi)=\rho\sin\varphi\\
	\end{cases}$$
	$$\left|\frac{D(x,y)}{D(\rho,\varphi)}\right|=|\rho|$$
	\par 因此一般选取$\rho\geqslant0$,从而使得$|\det\textbf{\textit{J}}_{\boldsymbol{\varphi}}(\rho,\varphi)|=\rho$。这样就从直角坐标系变换到了极坐标系。我们有
	$$\iint\limits_{D_2}f(x,y)\mathrm{d}x\mathrm{d}y=
	\iint\limits_{D_1}f(\rho\cos\varphi,\rho\sin\varphi)\rho\mathrm{d}\rho\mathrm{d}\varphi$$
	\par 如果$D_1$的形式为
	$$D_1=\{(\rho,\varphi)|\alpha\leqslant\varphi\leqslant\beta,\quad\rho_1(\varphi)\leqslant\rho\leqslant\rho_2(\varphi)\}$$
	\par 那么
	$$\iint\limits_{D_2}f(x,y)\mathrm{d}x\mathrm{d}y=
	\int_{\alpha}^{\beta}\mathrm{d}\varphi\int_{\rho_1(\varphi)}^{\rho_2(\varphi)}f(\rho\cos\varphi,\rho\sin\varphi)\rho\mathrm{d}\rho$$
	\par 如果$D_1$的形式为
	$$D_1=\{(\rho,\varphi)|a\leqslant\rho\leqslant b,\quad\varphi_1(\rho)\leqslant\varphi\leqslant\varphi_2(\rho)\}$$
	\par 那么
	$$\iint\limits_{D_2}f(x,y)\mathrm{d}x\mathrm{d}y=
	\int_{a}^{b}\mathrm{d}\rho\int_{\varphi_1(\rho)}^{\varphi_2(\rho)}f(\rho\cos\varphi,\rho\sin\varphi)\rho\mathrm{d}\varphi$$
	\par 对于椭圆,可能会涉及到伸缩后的极坐标变换$(a\geqslant0,b\geqslant0)$
	$$\begin{cases}
	x=\varphi_1(\rho,\varphi)=a\rho\cos\varphi\\
	y=\varphi_2(\rho,\varphi)=b\rho\sin\varphi\\
	\end{cases}$$
	\par 此时相应有
	$$\left|\frac{D(x,y)}{D(\rho,\varphi)}\right|=|ab\rho|$$
	$$\iint\limits_{D_2}f(x,y)\mathrm{d}x\mathrm{d}y=
	ab\iint\limits_{D_1}f(\rho\cos\varphi,\rho\sin\varphi)\rho\mathrm{d}\rho\mathrm{d}\varphi$$
	\subsection{其他计算方法}
	\par 二重积分的计算还可以用一些特殊方法。
	\par \textbf{利用对称性}\quad 如果积分区域$\Omega$关于$x$轴对称,则若$f(x,-y)=-f(x,y)$,则
	$$\iint\limits_{\Omega}f(x,y)\mathrm{d}x\mathrm{d}y=0$$
	\par 若$f(x,-y)=f(x,y)$,则
	$$\iint\limits_{\Omega}f(x,y)\mathrm{d}x\mathrm{d}y=2\iint\limits_{\Omega,half}f(x,y)\mathrm{d}x\mathrm{d}y$$
	\par 若区域关于$y$轴对称,则可以推出类似的结论。若区域关于原点对称,且$f(-x,-y)=-f(x,y)$,则
	$$\iint\limits_{\Omega}f(x,y)\mathrm{d}x\mathrm{d}y=0$$
	\par 有时交换$f(x,y)$中的$x$和$y$,也是一种求解的方法。
	\par \textbf{作特殊变换}\quad 除了极坐标变换外,也可以对$(x,y)$作其他变换。例如
	$$\begin{cases}
	x=\varphi_1(u,v)\\
	y=\varphi_2(u,v)\\
	\end{cases}$$
	$$\begin{cases}
	u=\phi_1(x,y)\\
	v=\phi_2(x,y)\\
	\end{cases}$$
	\par 如果积分区域是由这些曲线围成的不规则区域,可以尝试作上述变换,使其变成规则的区域甚至矩形区域。
	\par 但是进行坐标变换时要注意:(1)在计算Jacobi行列式时,注意是否取倒数;(2)在进行变换时,Jacobi行列式需要取绝对值。
	\section{三重积分的计算}
	\subsection{三重积分的一般计算方法}
	\par \textbf{X-Y型区域}\quad 设积分区域为$\Omega$,在$xOy$平面的投影为$D_{xy}$,如果$D_{xy}$是Jordan可测的,连续函数$z_1$,$z_2$满足:
	对于$\forall(x,y)\in D_{xy}$,都有$z_1(x,y)\leqslant z_2(x,y)$,则区域$\Omega$是Jordan可测的,且可以表示为
	$$\Omega=\{(x,y,z)|(x,y)\in D_{xy},\quad z_1(x,y)\leqslant z\leqslant z_2(x,y)\}$$
	\par 这时对于$\forall f\in C$,$f:\Omega\rightarrow\mathbb{R}$,有
	$$\iiint\limits_{\Omega}f(x,y,z)\mathrm{d}x\mathrm{d}y\mathrm{d}z=
	\iint\limits_{D_{xy}}\mathrm{d}x\mathrm{d}y\int_{z_1(x,y)}^{z_2(x,y)}f(x,y,z)\mathrm{d}z$$
	\par \textbf{Y-Z型区域}\quad 仿照上面一种类型,我们有
	$$\Omega=\{(x,y,z)|(y,z)\in D_{yz},\quad x_1(y,z)\leqslant x\leqslant x_2(y,z)\}$$
	$$\iiint\limits_{\Omega}f(x,y,z)\mathrm{d}x\mathrm{d}y\mathrm{d}z=
	\iint\limits_{D_{yz}}\mathrm{d}y\mathrm{d}z\int_{x_1(y,z)}^{x_2(y,z)}f(x,y,z)\mathrm{d}x$$
	\par \textbf{X-Z型区域}\quad 仿照上面一种类型,我们有
	$$\Omega=\{(x,y,z)|(x,z)\in D_{xz},\quad y_1(x,z)\leqslant y\leqslant y_2(x,z)\}$$
	$$\iiint\limits_{\Omega}f(x,y,z)\mathrm{d}x\mathrm{d}y\mathrm{d}z=
	\iint\limits_{D_{xz}}\mathrm{d}x\mathrm{d}z\int_{y_1(x,z)}^{y_2(x,z)}f(x,y,z)\mathrm{d}y$$
	\subsection{柱坐标与球坐标变换}
	\par \textbf{柱坐标变换}\quad 我们直接考虑伸缩后的广义柱坐标变换,$D_1=\{(\rho,\varphi,h)\}$,$D_2=\{(x,y,z)\}$,$a\geqslant0$,$b\geqslant0$,作变换
	$$\begin{cases}
	x=a\rho\cos\varphi\\
	y=b\rho\sin\varphi \quad & \quad (\rho\geqslant0,\quad 0\leqslant\varphi<2\pi)\\
	z=h\\
	\end{cases}$$
	\par 这时我们有
	$$\left|\frac{D(x,y,z)}{D(\rho,\varphi,h)}\right|=ab\rho$$
	因此
	$$\iiint\limits_{D_2}f(x,y,z)\mathrm{d}x\mathrm{d}y\mathrm{d}z=
	\iiint\limits_{D_1}f(a\rho\cos\varphi,b\rho\sin\varphi,h)ab\rho\mathrm{d}\rho\mathrm{d}\varphi\mathrm{d}h$$
	\par \textbf{球坐标变换}\quad
	接下来考虑球坐标变换,$D_1=\{(\rho,\theta,\varphi)\}$,$D_2=\{(x,y,z)\}$。则
	$$\begin{cases}
	x=\rho\sin\theta\cos\varphi\\
	y=\rho\sin\theta\sin\varphi\quad &  \quad(\rho\geqslant0,\quad0\leqslant\theta\leqslant\pi,\quad0\leqslant\varphi<2\pi)\\
	z=\rho\cos\theta\\
	\end{cases}$$
	$$\left|\frac{D(x,y,z)}{D(\rho,\theta,\varphi)}\right|=\rho^2\sin\theta$$
	\par 那么
	$$\iiint\limits_{D_2}f(x,y,z)\mathrm{d}x\mathrm{d}y\mathrm{d}z=
	\iiint\limits_{D_1}f(\rho\sin\theta\cos\varphi,\rho\sin\theta\sin\varphi,\rho\cos\theta)\rho^2\sin\theta\mathrm{d}\rho\mathrm{d}\theta\mathrm{d}\varphi$$
	\section{重积分的应用}
	\subsection{计算面积}
	二重积分可以计算平面图形的面积:在$xOy$平面上$D$区域的面积为
	$$S=|D|=\iint\limits_{D}\mathrm{d}x\mathrm{d}y$$
	\par 二重积分可以计算空间曲面的面积。假设空间曲面是以参数方程的形式定义的
	$$\begin{cases}
	x=x(u,v)\\
	y=y(u,v)\quad & \quad(u,v)\in D\\
	z=z(u,v)\\
	\end{cases}$$
	\par 如果$D$是Jordan可测的,$x$,$y$,$z$是连续函数,则$r(\textbf{\textit{J}}(x,y,z))=2$。其在两个方向的切向量为
	$$\textbf{\textit{T}}_u=\left(\frac{\partial x}{\partial u},\frac{\partial y}{\partial u},\frac{\partial z}{\partial u}\right)$$
	$$\textbf{\textit{T}}_v=\left(\frac{\partial x}{\partial v},\frac{\partial y}{\partial v},\frac{\partial z}{\partial v}\right)$$
	\par 两个切向量构成平行四边形的面积就是面积微元的大小,即$||\textbf{\textit{T}}_u\times \textbf{\textit{T}}_v||$。下面我们计算它的值,展开并化简可以得到
	$$\textbf{\textit{T}}_u\times \textbf{\textit{T}}_v=\left(\frac{D(y,z)}{D(u,v)},\frac{D(z,x)}{D(u,v)},\frac{D(x,y)}{D(u,v)}\right)$$
	$$||\textbf{\textit{T}}_u\times\textbf{\textit{T}}_v||^2=||\textbf{\textit{T}}_u||^2||\textbf{\textit{T}}_v||^2-(\textbf{\textit{T}}_u\cdot \textbf{\textit{T}}_v)^2$$
	\par 我们记
	$$E=\textbf{\textit{T}}_u\cdot \textbf{\textit{T}}_u$$
	$$G=\textbf{\textit{T}}_v\cdot \textbf{\textit{T}}_v$$
	$$F=\textbf{\textit{T}}_u\cdot \textbf{\textit{T}}_v$$
	\par 则面积微元
	$$\mathrm{d}\sigma=\sqrt{EG-F^2}\mathrm{d}u\mathrm{d}v$$
	\par 因此曲面的面积为
	$$S=\iint\limits_{D}\sqrt{EG-F^2}\mathrm{d}u\mathrm{d}v$$
	\par 如果曲面是按照$z=z(x,y)$的形式给出,代入上面的公式,我们有
	$$\begin{cases}
	x=u\\
	y=v\\
	z=z(u,v)\\
	\end{cases}$$
	$$\textbf{\textit{T}}_u=\left(1,0,\frac{\partial z}{\partial x}\right)$$
	$$\textbf{\textit{T}}_v=\left(0,1,\frac{\partial z}{\partial y}\right)$$
	$$E=1+\left(\frac{\partial z}{\partial x}\right)^2$$
	$$G=1+\left(\frac{\partial z}{\partial y}\right)^2$$
	$$F=\left(\frac{\partial z}{\partial x}\right)\left(\frac{\partial z}{\partial y}\right)$$
	$$\sqrt{EG-F^2}=\sqrt{1+\left(\frac{\partial z}{\partial x}\right)^2+\left(\frac{\partial z}{\partial y}\right)^2}$$
	\par 因此,曲面的面积为
	$$S=\iint\limits_{D}\sqrt{1+\left(\frac{\partial z}{\partial x}\right)^2+\left(\frac{\partial z}{\partial y}\right)^2}\mathrm{d}x\mathrm{d}y$$
	\par 对于$x=x(y,z)$和$y=y(x,z)$,情形是类似的。
	\subsection{计算体积}
	二重积分可以计算曲顶柱体的体积:在$xOy$平面上的$D$区域,$z=0$和$z=f(x,y)$之间的部分构成曲顶柱体,其体积为
	$$V=\iint\limits_{D}f(x,y)\mathrm{d}x\mathrm{d}y$$
	\par 三重积分可以计算封闭几何体的体积:对于三维区域$\Omega$,其体积为
	$$V=|\Omega|=\iiint\limits_{\Omega}\mathrm{d}x\mathrm{d}y\mathrm{d}z$$
	\subsection{计算质心}
	在三维区域$\Omega$中,有密度分布$\rho(\textbf{\textit{X}})$,则这个几何体的质量为
	$$M=\iiint\limits_{\Omega}\rho(\textbf{\textit{X}})\mathrm{d}V$$
	\par 设其\textbf{质心}为$(\bar{x},\bar{y},\bar{z})$,则有
	$$\bar{x}=\frac{1}{M}\iiint\limits_{\Omega}x\rho(\textbf{\textit{X}})\mathrm{d}V$$
	$$\bar{y}=\frac{1}{M}\iiint\limits_{\Omega}y\rho(\textbf{\textit{X}})\mathrm{d}V$$
	$$\bar{z}=\frac{1}{M}\iiint\limits_{\Omega}z\rho(\textbf{\textit{X}})\mathrm{d}V$$
	\par 若令$\rho(\textbf{\textit{X}})\equiv1$,则得到的点$(\bar{x},\bar{y},\bar{z})$称为几何体的\textbf{几何中心}。
	\par 注意以下两点:(1)这种计算方法可以推广到任意维度;
	(2)这种方法只适用于直角坐标系,如果直接用$(\rho,\varphi)$代替$(x,y)$,会得到错误的结果。
	\chapter{曲线与曲面积分}
	\section{第一型曲线积分}
	\subsection{第一型曲线积分的定义}
	我们称$L\subset\mathbb{R}^{3}$为一条空间曲线,$L$开始于$A$,结束于$B$。定义映射$f:L\rightarrow\mathbb{R}$。对曲线$L$进行分割,形成若干弧$P_0P_1$,$P_1P_2$,...,$P_{n-1}P_{n}$,且$A=P_0$,$B=P_n$,将这个分割称为$\tau$。定义$\tau$的步长$$d=\max_i|P_{i-1}P_i|$$\par
	在弧$P_{i-1}P_{i}$上任意选取一点$P_{i}^*$,对应有函数值$f(P_i^*)$。
	若极限
	$$I=\lim_{d\rightarrow 0}\sum_{i=1}^{n}f(P_i^*)|P_{i-1}P_{i}|$$存在,
	我们称极限值$I$为函数$f(x,y,z)$在曲线$L$上的\textbf{第一型曲线积分},记作
	$$\int\limits_L f(x,y,z)\mathrm{d}l\quad or\quad\int_{A}^{B}f(x,y,z)\mathrm{d}l$$
	\subsection{第一型曲线积分的性质}
	如果$L$分段光滑,$f$连续,则积分存在。\par
	曲线的长度可以表示为
	$$\int_{A}^{B}1\mathrm{d}l$$
	对于二维曲线$L\subset\mathbb{R}^{2}$,依然可以定义曲线积分
	$$\int\limits_{L}f(x,y)\mathrm{d}l$$\par
	它表示曲线$L$上位于$z=0$和$z=f(x,y)$之间的柱面的面积。\par
	对于被积函数,有界性、线性、保号保序性等依然存在。这里只介绍关于路径的性质:无向性。
	$$\int_{A}^{B}f(x,y,z)\mathrm{d}l=\int_{B}^{A}f(x,y,z)\mathrm{d}l$$
	\subsection{第一型曲线积分的计算}
	如果曲线$L$是按照参数方程的形式定义的
	$$\begin{cases}
	x=x(t)\\y=y(t) & t\in \left[\alpha,\beta\right]\\z=z(t)
	\end{cases}$$\par
	则	$$\mathrm{d}l=\sqrt{\left[x'(t)\right]^2+
	\left[y'(t)\right]^2+\left[z'(t)\right]^2}\mathrm{d}t$$\par
	因此我们有
	$$\int\limits_L f(x,y,z)\mathrm{d}l=
	\int_{\alpha}^{\beta}f\left(x(t),y(t),z(t)\right)\sqrt{\left[x'(t)\right]^2+
	\left[y'(t)\right]^2+\left[z'(t)\right]^2}\mathrm{d}t$$
	\section{第一型曲面积分}
	\subsection{第一型曲面积分的定义}
	我们定义空间曲面$S\subset\mathbb{R}^3$,并定义上面的映射$f:S\rightarrow\mathbb{R}$。
现在我们可以对将曲面分为更小的曲面$S_1$,$S_2$,\ldots,$S_n$。在每一个小曲面上取点$X_i\in S_i$,
则每个点对应函数值$f(X_i)$。我们定义这个分割的直径$$\mathbb{D}=\max_{1\leq i\leq n}d(S_i)$$\par
	我们在前面定义了直径的概念:
	$$d(S)=\sup_{i\neq j,X_i,X_j\in S}||X_i-X_j||$$\par
	如果极限
	$$I=\lim_{\mathbb{D}\rightarrow 0}\sum_{i=1}^{n}f(X_i)|S_i|$$
	存在,那么我们将$I$称为函数$f(x,y,z)$在曲面$S$上的\textbf{第一型曲面积分},记作
	$$I=\iint\limits_S f(x,y,z)\mathrm{d}\sigma$$
	\subsection{第一型曲面积分的性质}
	如果$S$分片光滑,$f$连续,则积分存在。\par
	曲面S的面积可以表示为
	$$\iint\limits_{S}1\mathrm{d}\sigma$$\par
	对于二维空间由曲线围成的平面,仍然可以定义第一型曲面积分,并且
	$$\iint\limits_{S}f(x,y)\mathrm{d}\sigma=\iint\limits_{S}f(x,y)\mathrm{d}x\mathrm{d}y$$\par
	我们已经知道,这个积分表示平面$S$上位于$z=0$和$z=f(x,y)$之间的曲顶柱体的体积。
	\subsection{第一型曲面积分的计算}
	假设曲面S是用参数方程的形式定义的\\
	$$\begin{cases}
	x=x(u,v)\\y=y(u,v)&\qquad(u,v)\in\Omega\\z=z(u,v)
	\end{cases}$$\par
	我们定义
	$$\textbf{\textit{T}}_{u}=\left(\frac{\partial x}{\partial u}, \frac{\partial y}{\partial u},
	\frac{\partial z}{\partial u}\right)^{T}$$
	$$\textbf{\textit{T}}_{v}=\left(\frac{\partial x}{\partial v}, \frac{\partial y}{\partial v},
	\frac{\partial z}{\partial v}\right)^{T}$$
	$$E=\textbf{\textit{T}}_{u}\cdot\textbf{\textit{T}}_{u}$$
	$$G=\textbf{\textit{T}}_{v}\cdot\textbf{\textit{T}}_{v}$$
	$$F=\textbf{\textit{T}}_{u}\cdot\textbf{\textit{T}}_{v}$$\par
	则面积微元可以表示为
	$$\mathrm{d}\sigma=\sqrt{EG-F^2}\mathrm{d}u\mathrm{d}v$$
	那么
	$$\iint\limits_{S}f(x,y,z)\mathrm{d}\sigma=
	\iint\limits_{\Omega}f\left(x(u,v),y(u,v),z(u,v)\right)\sqrt{EG-F^2}\mathrm{d}u\mathrm{d}v$$\par
	特别的,如果曲面直接用$z=f(x,y)$的形式表示,,且$(x,y)\in\Omega$则
	$$\iint\limits_{S}f(x,y,z)\mathrm{d}\sigma=
	\iint\limits_{\Omega}f\left(x,y,z(x,y)\right)\sqrt{1+\left(\frac{\partial z}{\partial x}\right)^2+
	\left(\frac{\partial z}{\partial y}\right)^2}\mathrm{d}x\mathrm{d}y$$
\section{第二型曲线积分}
\subsection{第二型曲线积分的定义}
	我们定义空间曲线$L\subset\mathbb{R}^3$,起点为$A$,终点为$B$。定义向量值函数$\textbf{\textit{F}}=(F_1,F_2,F_3)^{T}$,
	$\textbf{\textit{F}}:L\rightarrow\mathbb{R}^3$。现在对曲线$L$进行分割,得到弧线$P_0P_1$,$P_1P_2$,...,$P_{n-1}P_{n}$。在每一个弧线上取点$\textbf{\textit{X}}_i^*$,得到函数值$\textbf{\textit{F}}(\textbf{\textit{X}}_i^*)$。\par
	计算弧线起点终点构成的向量与函数值的内积
	$$\overrightarrow{P_{i-1}P_{i}}\cdot\textbf{\textit{F}}(\textbf{\textit{X}}_i^*)=
	F_1(\textbf{\textit{X}}_i^*)(x_i-x_{i-1})+F_2(\textbf{\textit{X}}_i^*)(y_i-y_{i-1})+
	F_3(\textbf{\textit{X}}_i^*)(z_i-z_{i-1})$$\par
	定义分割的步长
	$$d=\max_{i}|P_{i-1}P_{i}|$$\par
	如果极限
	$$I=\lim_{d\rightarrow 0}\sum_{i=1}^{n}\overrightarrow{P_{i-1}P_{i}}\cdot\textbf{\textit{F}}(\textbf{\textit{X}}_i^*)$$
	存在,则将极限值称为函数$F$沿路径$L$从$A$到$B$的\textbf{第二型曲线积分},若定义$\textbf{\textit{l}}=(x,y,z)$,则积分可以记作
	$$I=\int_{A}^{B}\textbf{\textit{F}}(\textbf{\textit{l}})\cdot\mathrm{d}\textbf{\textit{l}}$$。\par
	如果路径$L$是闭合的,则其起点和终点可以任意选取,且将其逆时针环绕一周的方向定义为\textbf{正方向}。此时的积分可以记作
	$$\oint\limits_{L^{+}}\textbf{\textit{F}}(\textbf{\textit{l}})\cdot\mathrm{d}\textbf{\textit{l}}$$
	\subsection{第二型曲线积分的性质}
	关于被积函数的性质,这里不再赘述。下面是几条关于积分路径的性质。\par
	积分路径的有向性:
	$$\int_{A}^{B}\textbf{\textit{F}}(\textbf{\textit{l}})\cdot\mathrm{d}\textbf{\textit{l}}=
	-\int_{B}^{A}\textbf{\textit{F}}(\textbf{\textit{l}})\cdot\mathrm{d}\textbf{\textit{l}}$$\par
	积分路径的可加性:
	$$\int_{A}^{B}\textbf{\textit{F}}(\textbf{\textit{l}})\cdot\mathrm{d}\textbf{\textit{l}}=
	\int_{A}^{C}\textbf{\textit{F}}(\textbf{\textit{l}})\cdot\mathrm{d}\textbf{\textit{l}}+
	\int_{C}^{B}\textbf{\textit{F}}(\textbf{\textit{l}})\cdot\mathrm{d}\textbf{\textit{l}}$$\par
	环形积分路径可加性:
	$$\oint\limits_{L^+}\textbf{\textit{F}}(\textbf{\textit{l}})\cdot\mathrm{d}\textbf{\textit{l}}=
	\oint\limits_{L_1^+}\textbf{\textit{F}}(\textbf{\textit{l}})\cdot\mathrm{d}\textbf{\textit{l}}+
	\oint\limits_{L_2^+}\textbf{\textit{F}}(\textbf{\textit{l}})\cdot\mathrm{d}\textbf{\textit{l}}$$
	\subsection{第二型曲线积分的计算}
	由于$F=(F_1,F_2,F_3)$,$\textbf{\textit{l}}=(x,y,z)$,可以对积分进行展开:
	$$\int\limits_{L}\textbf{\textit{F}}(\textbf{\textit{l}})\cdot\mathrm{d}\textbf{\textit{l}}=
	\int\limits_{L}\left(F_1(x,y,z),F_2(x,y,z),F_3(x,y,z)\right)\cdot(\mathrm{d}x,\mathrm{d}y,\mathrm{d}z)$$\par
	所以
	$$\int\limits_{L}\textbf{\textit{F}}(\textbf{\textit{l}})\cdot\mathrm{d}\textbf{\textit{l}}=
	\int\limits_{L}F_1(x,y,z)\mathrm{d}x+F_2(x,y,z)\mathrm{d}y+F_3(x,y,z)\mathrm{d}z$$
	第二型曲线积分也经常表示成上面的形式。\par
	如果曲线是按照参数方程的形式定义的
	$$\begin{cases}
	x=x(t)\\y=y(t)&\qquad\qquad t\in[\alpha,\beta]\\z=z(t)
	\end{cases}$$\par
	那么积分就可以表示为
	$$\int_{\alpha}^{\beta}F_1(x(t),y(t),z(t))x'(t)\mathrm{d}t+
	F_2(x(t),y(t),z(t))y'(t)\mathrm{d}t+F_3(x(t),y(t),z(t))z'(t)\mathrm{d}t$$
	\subsection{第一型曲线积分与第二型曲线积分的关系}
	对于空间曲线$\textbf{\textit{l}}(t)=(x(t),y(t),z(t))$,在某一点处的单位切向量为
	$$\boldsymbol{\tau}^{0}(t)=\frac{\boldsymbol{\tau}(t)}{||\boldsymbol{\tau}(t)||}=(\cos\alpha(t),\cos\beta(t),\cos\gamma(t))$$
	$$\boldsymbol{\tau}(t)=(x'(t),y'(t),z'(t))$$
	$$||\boldsymbol{\tau}(t)||=\sqrt{[x'(t)]^2+[y'(t)]^2+[z'(t)]^2}=\frac{\mathrm{d}l}{\mathrm{d}t}$$\par
	因此
	$$\mathrm{d}x(t)=x'(t)\mathrm{d}t=\cos\alpha(t)\textrm{d}l$$
	$$\mathrm{d}y(t)=y'(t)\mathrm{d}t=\cos\beta(t)\textrm{d}l$$
	$$\mathrm{d}z(t)=z'(t)\mathrm{d}t=\cos\gamma(t)\textrm{d}l$$\par
	也就是
	$$\mathrm{d}\textbf{\textit{l}}=\boldsymbol{\tau}^{0}\mathrm{d}l$$\par
	所以
	$$\int_{A}^{B}F(\textbf{\textit{l}})\cdot\mathrm{d}\textbf{\textit{l}}=\int_{A}^{B}F(\textbf{\textit{l}})\cdot\boldsymbol{\tau}^0\mathrm{d}l$$
	\section{第二型曲面积分}
	\subsection{第二型曲面积分的定义}
	\par 设$S\subset\mathbb{R}$是光滑连通的曲面,$\textbf{\textit{r}}=(x,y,z)$,曲面的参数方程为
	$$\begin{cases}
	x=x(u,v)\\
	y=y(u,v) & \qquad (u,v)\in D\\
	z=z(u,v)\\
	\end{cases}$$
	\par 那么就可以得到两个法向量
	$$\textbf{\textit{n}}_{+}=+\textbf{\textit{T}}_u\times\textbf{\textit{T}}_v$$
	$$\textbf{\textit{n}}_{-}=-\textbf{\textit{T}}_u\times\textbf{\textit{T}}_v$$
	\par 设$\textbf{\textit{P}}_0\in S$,则过这一点有法向量$\textbf{\textit{n}}$,如果在过$\textbf{\textit{P}}_0$的任意曲线上,$\textbf{\textit{n}}$是连续的,则称曲面$S$是\textbf{可定向的},否则是\textbf{不可定向的}。例如莫比乌斯环就是不可定向的。对于连通光滑曲面,其可定向的充要条件是法向量永不为零。
	\par 在定义了可定向后,就可以定义第二型曲面积分。假设开集$\Omega\in\mathbb{R}^3$,$S\subset\Omega$,选取$S$的正方向$S^+$。上面定义有向量值函数$\textbf{\textit{F}}=(P,Q,R)$。将$S$分为若干小曲面$S_1$,$\cdots$,$S_n$,并取点$\textbf{\textit{X}}_i$,定义有向面积
	$$\textbf{\textit{S}}_i=|S_i|\textbf{\textit{n}}^0(\textbf{\textit{X}}_i)$$
	\par 进一步可以定义Riemann和
	$$\sum_{i=1}^{n}F(\textbf{\textit{X}}_i)\cdot\textbf{\textit{S}}_i$$
	\par 定义步长
	$$D=\max_{1\leqslant i\leqslant n}d(S_i)$$
	\par 如果极限
	$$I=\sum_{i=1}^{n}F(\textbf{\textit{X}}_i)\cdot\textbf{\textit{S}}_i$$
	存在,则称$I$为$F$在$S$上的\textbf{第二型曲面积分},记为
	$$\iint\limits_{S^+}\textbf{\textit{F}}(x,y,z)\cdot\mathrm{d}\textbf{\textit{S}}$$
	\par 如果$S$是封闭曲面,往往取从内向外为正方向,并把积分记作
	$$\varoiint\limits_{S^+}\textbf{\textit{F}}(x,y,z)\cdot\mathrm{d}\textbf{\textit{S}}$$
	\subsection{第二型曲面积分的性质}
	\par 如果$S$分片光滑,$F$分段连续,则积分存在。同样,下面是关于积分区域的性质。
	\par \textbf{有向性}\quad 我们有
	$$\iint\limits_{S^+}\textbf{\textit{F}}(x,y,z)\cdot\mathrm{d}\textbf{\textit{S}}=
	-\iint\limits_{S^-}\textbf{\textit{F}}(x,y,z)\cdot\mathrm{d}\textbf{\textit{S}}$$
	\par \textbf{可加性}\quad 如果$S$由$S_1$,$S_2$组成,后者由$S$的定向诱导,则
	$$\iint\limits_{S^+}\textbf{\textit{F}}(x,y,z)\cdot\mathrm{d}\textbf{\textit{S}}=
	\iint\limits_{S^+_1}\textbf{\textit{F}}(x,y,z)\cdot\mathrm{d}\textbf{\textit{S}}+
	\iint\limits_{S^+_2}\textbf{\textit{F}}(x,y,z)\cdot\mathrm{d}\textbf{\textit{S}}$$
	\subsection{第一型曲面积分和第二型曲面积分的联系}
	\par 从前文中,我们可以看到
	$$\mathrm{d}\textbf{\textit{S}}=\textbf{\textit{n}}^0\mathrm{d}S$$
	\par 由此可以得到两类曲面积分的联系
	$$\iint\limits_{S^+}\textbf{\textit{F}}\cdot\mathrm{d}\textbf{\textit{S}}=
	\iint\limits_{S}(\textbf{\textit{F}}\cdot\textbf{\textit{n}}^0)\mathrm{d}S$$
	\par 我们记
	$$\textbf{\textit{n}}^0=(\cos\alpha,\cos\beta,\cos\gamma)$$
	\par 则根据两类积分的关系,$\textbf{\textit{F}}=(P,Q,R)$,我们有
	$$\iint\limits_{S^+}\textbf{\textit{F}}\cdot\mathrm{d}\textbf{\textit{S}}=
	\iint\limits_{S}\left(P(x,y,z)\cos\alpha+Q(x,y,z)\cos\beta+R(x,y,z)\cos\gamma\right)\mathrm{d}S$$
	\par 进一步,我们记
	$$\mathrm{d}y\land\mathrm{d}z=\cos\alpha\mathrm{d}S$$
	$$\mathrm{d}z\land\mathrm{d}x=\cos\beta\mathrm{d}S$$
	$$\mathrm{d}x\land\mathrm{d}y=\cos\gamma\mathrm{d}S$$
	\par 那么积分可以表示为
	$$\iint\limits_{S}P(x,y,z)\mathrm{d}y\land\mathrm{d}z+
	Q(x,y,z)\mathrm{d}z\land\mathrm{d}x+R(x,y,z)\mathrm{d}x\land\mathrm{d}y$$
	\subsection{第二型曲面积分的计算}
	\par 我们已经知道其法向量的求法
	$$\textbf{\textit{n}}=\textbf{\textit{T}}_u\times\textbf{\textit{T}}_v=\left(\frac{D(y,z)}{D(u,v)},\frac{D(z,x)}{D(u,v)},\frac{D(x,y)}{D(u,v)}\right)$$
	$$||\textbf{\textit{n}}||=\sqrt{EG-F^2}$$
	$$\mathrm{d}\textbf{\textit{S}}=\textbf{\textit{n}}^0\mathrm{d}S=\textbf{\textit{n}}^0\sqrt{EG-F^2}\mathrm{d}u\mathrm{d}v=\textbf{\textit{n}}^0||\textbf{\textit{n}}||\mathrm{d}u\mathrm{d}v=\textbf{\textit{n}}\mathrm{d}u\mathrm{d}v$$
	\par 上面的式子成立的条件是法向量与曲面的方向一致。如果相反,需要在结果前加上负号。这样我们就有
	$$\iint\limits_{S^+}\textbf{\textit{F}}\cdot\mathrm{d}\textbf{\textit{S}}=\pm\iint\limits_{S}(\textbf{\textit{F}}\cdot\textbf{\textit{n}}\mathrm{d}u)\mathrm{d}v$$
	\par 进一步展开,我们有
	$$\pm\textbf{\textit{F}}\cdot\textbf{\textit{n}}=\pm\left[P(u,v)\frac{D(y,z)}{D(u,v)}+Q(u,v)\frac{D(z,x)}{D(u,v)}+R(u,v)\frac{D(z,y)}{D(u,v)}\right]$$
	\par 也可以写成行列式的形式
	$$\pm\textbf{\textit{F}}\cdot\textbf{\textit{n}}=\pm
	\begin{vmatrix}
	P & Q & R\\
	\frac{\partial x}{\partial u} & \frac{\partial y}{\partial u} & \frac{\partial z}{\partial u}\\
	\frac{\partial x}{\partial v} & \frac{\partial y}{\partial v} & \frac{\partial z}{\partial v}\\
	\end{vmatrix}$$
	\par 综上所述,我们有
	$$\iint\limits_{S^+}P(x,y,z)\mathrm{d}y\land\mathrm{d}z=
	\iint\limits_{D}P(u,v)\left(\pm\frac{D(y,z)}{D(u,v)}\right)\mathrm{d}u\mathrm{d}v$$
	$$\iint\limits_{S^+}Q(x,y,z)\mathrm{d}z\land\mathrm{d}x=
	\iint\limits_{D}Q(u,v)\left(\pm\frac{D(z,x)}{D(u,v)}\right)\mathrm{d}u\mathrm{d}v$$
	$$\iint\limits_{S^+}R(x,y,z)\mathrm{d}x\land\mathrm{d}y=
	\iint\limits_{D}R(u,v)\left(\pm\frac{D(x,y)}{D(u,v)}\right)\mathrm{d}u\mathrm{d}v$$
	\par 其中正负号的判据有下面两点:(1)如果法向量与曲面定向同向,取正号,相反则取负号;(2)Jacobi行列式一项的符号应该与法向量相应分量(方向角余弦)的符号相同。
	\section{Green公式\quad Gauss公式\quad Stokes公式}
	\subsection{Green公式}
	\par 我们定义$\mathbb{R}^2$中的\textbf{单连通集合}为:对于$D\subset\mathbb{R}^2$,如果$D$中任意封闭曲线包围的区域仍然在$D$中,称$D$是单连通的;否则是复连通的。
	\par 假设$\Omega$是单连通的有界闭区域,边界$\partial\Omega$分段光滑,设$\textbf{\textit{n}}^0$是边界的单位外法向量。定义上面的向量值函数
	$$\textbf{\textit{F}}=(F_1,F_2)\in C^{(1)}(\Omega;\mathbb{R}^2)$$
	\par 那么
	$$\oint\limits_{\partial\Omega^+}(\textbf{\textit{F}}\cdot\textbf{\textit{n}}^0)\mathrm{d}l=
	\iint\limits_{\Omega}\left(\frac{\partial F_1}{\partial x}(x,y)+\frac{\partial F_2}{\partial y}(x,y)\right)\mathrm{d}x\mathrm{d}y$$
	\par 假设在一点$P$处的切向量为$\boldsymbol{\tau}^0=(\cos\alpha,\sin\alpha)$,则外法向量为$n^0=(\sin\alpha,-\cos\alpha)$,我们有$\cos\alpha\mathrm{d}l=\mathrm{d}x$,$\sin\alpha\mathrm{d}l=\mathrm{d}y$,那么我们可以得到
	$$\oint\limits_{\partial\Omega^+}F_1\mathrm{d}y-F_2\mathrm{d}x=
	\iint\limits_{\Omega}\left(\frac{\partial F_1}{\partial x}(x,y)+\frac{\partial F_2}{\partial y}(x,y)\right)\mathrm{d}x\mathrm{d}y$$
	\par 我们对函数名称进行代换,可以得到另一种形式
	$$\oint\limits_{\partial\Omega^+}F_1\mathrm{d}x+F_2\mathrm{d}y=
	\iint\limits_{\Omega}\left(\frac{\partial F_2}{\partial x}(x,y)-\frac{\partial F_1}{\partial y}(x,y)\right)\mathrm{d}x\mathrm{d}y$$
	\par 将上面两种形式总结一下,我们可以得到下面的两个公式。这个公式可以拓宽到复连通区域,此时对于正方向的定义是:沿着正方向前进,区域一直在左手边。
	$$\oint\limits_{\partial\Omega^+}P\mathrm{d}y-Q\mathrm{d}x=
	\iint\limits_{\Omega}\left(\frac{\partial P}{\partial x}(x,y)+\frac{\partial Q}{\partial y}(x,y)\right)\mathrm{d}x\mathrm{d}y$$
	$$\oint\limits_{\partial\Omega^+}P\mathrm{d}x+Q\mathrm{d}y=
	\iint\limits_{\Omega}\left(\frac{\partial Q}{\partial x}(x,y)-\frac{\partial P}{\partial y}(x,y)\right)\mathrm{d}x\mathrm{d}y$$
	\subsection{Gauss公式}
	\par 设$\Omega$是$\mathbb{R}^3$中的有界闭区域,其边界$\partial\Omega$分片光滑可定向,并记外侧为正向,设
	$$\textbf{\textit{F}}=(P,Q,R)\in C^{(1)}(\Omega;\mathbb{R}^3)$$
	\par 那么
	$$\varoiint\limits_{\partial\Omega^+}\textbf{\textit{F}}\cdot\mathrm{d}\textbf{\textit{S}}=\iiint\limits_{\Omega}\left(\frac{\partial P}{\partial x}+\frac{\partial Q}{\partial y}+\frac{\partial R}{\partial z}\right)\mathrm{d}x\mathrm{d}y\mathrm{d}z$$
	\par Gauss公式与Green公式是很相似的,可以看成是Green公式的三维情形。
	$$\varoiint\limits_{\partial\Omega}(\textbf{\textit{F}}\cdot\textbf{\textit{n}}^0)\mathrm{d}S=\iiint\limits_{\Omega}\left(\frac{\partial P}{\partial x}+\frac{\partial Q}{\partial y}+\frac{\partial R}{\partial z}\right)\mathrm{d}x\mathrm{d}y\mathrm{d}z$$
	\subsection{Stokes公式}
	\par 假设$\Omega$是非空开集,其中$S$是分片光滑曲面,边界$\partial S$是分段光滑闭曲线,$S^+$与$\partial S^+$满足右手螺旋法则,设向量值函数$\textbf{\textit{F}}=(P,Q,R)\in C^{(1)}$,则
	$$\oint\limits_{\partial S^+}\textbf{\textit{F}}\cdot\mathrm{d}\textbf{\textit{l}}=\iint\limits_{S^+}\textbf{rot}\textbf{\textit{F}}\cdot\mathrm{d}\textbf{\textit{S}}$$
	\par 其中
	$$\textbf{rot}\textbf{\textit{F}}=\nabla\times\textbf{\textit{F}}=
	\begin{vmatrix}
	\textbf{\textit{i}} & \textbf{\textit{j}} & \textbf{\textit{k}} \\
	\partial_x & \partial_y & \partial_z \\
	P & Q & R\\
	\end{vmatrix}$$
	\section{场论初步}
	\subsection{梯度、散度与旋度}
	\par 在研究场时,经常用到nabla算子
	$$\nabla=\left(\frac{\partial}{\partial x},\frac{\partial}{\partial y},\frac{\partial}{\partial z}\right)$$
	\par 首先考虑标量场,设$\textbf{\textit{p}}=(x,y,z)$,定义在$\Omega\in\mathbb{R}^3$上的标量场
	$F=F(\textbf{\textit{p}})$实际上就是一个多元函数。对于标量场,我们经常用梯度来研究,并定义\textbf{梯度}
	$$\mathbf{grad}F=\nabla F=(\partial_x F,\partial_y F, \partial_z F)$$
	\par 其次考虑向量场,定义在$\Omega$上的向量场$\textbf{\textit{F}}=\textbf{\textit{F}}(\textbf{\textit{p}})=(P,Q,R)$实际就是一个向量值函数。我们定义其\textbf{散度}为
	$$\mathrm{div}\textbf{\textit{F}}=\nabla\cdot\textbf{\textit{F}}=\partial_x P+\partial_y Q+\partial_z R$$
	\par 同时定义其\textbf{旋度}为
	$$\textbf{rot}\textbf{\textit{F}}=\nabla\times\textbf{\textit{F}}=
	\begin{vmatrix}
	\textbf{\textit{i}} & \textbf{\textit{j}} & \textbf{\textit{k}}\\
	\partial_x & \partial_y & \partial_z \\
	P & Q & R\\
	\end{vmatrix}$$
	$$\textbf{rot}\textbf{\textit{F}}=(\partial_y R-\partial_z Q,\partial_z P-\partial_x R,\partial_x Q-\partial_z P)$$
	\par 借助这些符号,Gauss定理和Stokes定理可以表示为
	$$\varoiint\limits_{\partial S^+}\textit{\textbf{F}}\cdot\mathrm{d}\textbf{\textit{S}}=\iiint\limits_{S}\nabla\cdot\textbf{\textit{F}}\mathrm{d}x\mathrm{d}y\mathrm{d}z$$
	$$\oint\limits_{\partial S^+}\textbf{\textit{F}}\cdot\mathrm{d}\textbf{\textit{l}}=\iint\limits_{S^+}(\nabla\times\textbf{\textit{F}})\cdot\mathrm{d}\textbf{\textit{S}}$$
	\subsection{保守场}
	\par 保守场是场论中一种重要的场,保守场与一种具有特定性质的函数相联系。
	\par 首先考虑二维情况,对于二维空间的单连通开区域,上面定义有函数$\textbf{\textit{F}}$,下面四种描述是等价的。
	\par (1)对于区域内任意一点,都有
	$$\frac{\partial F_1}{\partial y}=\frac{\partial F_2}{\partial x}$$
	\par (2)对于区域内任意两点$A$,$B$,下面的第二型曲线积分只与两点的坐标有关,而与具体的路径无关,也就是
	$$\int\limits_{\forall L}\textbf{\textit{F}}\cdot\mathrm{d}\textbf{\textit{l}}\equiv C$$
	\par (3)对于区域内任意一个闭合路径,都有
	$$\oint\limits_{L^+}\textbf{\textit{F}}\cdot\mathrm{d}\textbf{\textit{l}}\equiv0$$
	\par (4)存在二元函数$U$,满足
	$$\mathrm{d}U=F_1\mathrm{d}x+F_2\mathrm{d}y$$
	也就是
	$$\frac{\partial U}{\partial x}=F_1$$
	$$\frac{\partial U}{\partial y}=F_2$$
	\par 满足上面的向量值函数,其第二型曲线积分与积分路径无关,而$U(\pm C)$称为微分表达式的原函数,此时从$A$到$B$的积分有下面的记法,并且满足
	$$\left.\int_{A}^{B}F\cdot\mathrm{d}L=U\right|_{A}^B$$
	\par 现在将情形推广到三维,此时我们有类似的四个等价表述,其中$\textbf{\textit{F}}=(P,Q,R)$
	\par (1)对于区域内任意一点,都有
	$$\nabla\times\textbf{\textit{F}}=0$$
	\par (2)对于区域内任意两点$A$,$B$,下面的第二型曲线积分只与两点的坐标有关,而与具体的路径无关,也就是
	$$\int\limits_{\forall L}\textbf{\textit{F}}\cdot\mathrm{d}\textbf{\textit{l}}\equiv C$$
	\par (3)对于区域内任意一个闭合路径,都有
	$$\oint\limits_{L^+}\textbf{\textit{F}}\cdot\mathrm{d}\textbf{\textit{l}}\equiv0$$
	\par (4)存在标量场$\phi$,满足
	$$\nabla\phi=\textbf{\textit{F}}$$
	也就是$\partial_x\phi=P$,$\partial_y\phi=Q$,$\partial_y\phi=R$。
	\par 满足上述四个条件的向量场称为\textbf{保守场}、\textbf{无旋场}或者\textbf{有势场},而$\phi$称为这个场的\textbf{势函数}。固定一点$A$,则$\textbf{\textit{p}}$点的势函数可以用下面的方法得到
	$$\phi(\textbf{\textit{p}})=\int_{A}^{\textbf{\textit{p}}}F\cdot\mathrm{d}\textbf{\textit{l}}$$
	\par 这就是有势场和势函数的互相转换关系
	$$\begin{cases}
		\textbf{\textit{F}}=\nabla\phi\\
		\phi=\int\textbf{\textit{F}}\cdot\mathrm{d}\textbf{\textit{l}}\\
	\end{cases}$$
	

















\end{document}